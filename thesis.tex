%volby: 
% male × female
% czech × english (zatím funguje jen czech)
% a studijní program / obor 
% is_bc (nejvíc odladěno)
% api_bc
% api_ing
% edu_bc
% edu_ing

\documentclass[male,czech,api_ing]{thesis}
\usepackage{ifthen}

\usepackage{amsmath,amssymb}
\usepackage{graphics}
\usepackage{color}
\usepackage{array}
\usepackage{longtable}
\usepackage{afterpage}
\usepackage{microtype}  % přesnější typografie
\usepackage{xcolor}
\usepackage{graphicx}

\graphicspath{ {./images/} }

\usepackage{float}

% workaround for imcompatibility of czech babel and biblatex

\iftutex
\else
\usepackage{etoolbox}
\makeatletter
\newcommand\my@hyphen{-}
\newcommand\my@apostroph{'}
\patchcmd\select@language{-}{\my@hyphen }{}{}
\patchcmd\select@language{'}{\my@apostroph }{}{}
\makeatother
\fi

% fonty lze měnit (detaily viz sekce fonty)
\iftutex
	\usepackage{fontspec}  % nastavení fontů pro LuaLaTeX a XeLaTeX
	\setmainfont{Libertinus Serif}
	\setsansfont{Libertinus Sans}
	\setmonofont[Scale=MatchLowercase]{Libertinus Mono}
	\usepackage{unicode-math}
	\setmathfont{Libertinus Math}
\else
	\usepackage[utf8]{inputenc} % nastavení pro PDF LaTeX
	\usepackage[T1]{fontenc}
	\usepackage{libertinus}
	\renewcommand{\ttdefault}{pxtt}
\fi

\usepackage[style=iso-numeric,shortnumeration=true]{biblatex}
\addbibresource{thesis.bib}

\usepackage{csquotes} % uvozovky

% sazba ukázek kódu 

\usepackage{listings}
\renewcommand{\lstlistingname}{Ukázka kódu}

\lstdefinestyle{DOS}
{
    basicstyle=\scriptsize\color{black}\ttfamily,
    breaklines=true,
}

% ukázka pro nastavení balíku listings pro sazbu ukázek zdrojových kódů
\lstset{
  language=Python,                % the language of the code
  basicstyle=\small\ttfamily,    
  backgroundcolor=\color{white},   % choose the background color. You must add \usepackage{color}
  showspaces=false,                % show spaces adding particular underscores
  showstringspaces=true,           % underline spaces within strings
  showtabs=false,                  % show tabs within strings adding particular underscores
  frame=single,                    % adds a frame around the code
  tabsize=3,                       % sets default tabsize to 2 spaces
  breaklines=true,                 % sets automatic line breaking
  breakatwhitespace=false,         % sets if automatic breaks should only happen at whitespace
  keywordstyle=\bfseries,          % keyword style
  commentstyle=\rmfamily,       % comment style
  stringstyle=\itshape\color,   % string literal style
}

\definecolor{bluekeywords}{rgb}{0.13,0.13,1}
\definecolor{greencomments}{rgb}{0,0.5,0}
\definecolor{redstrings}{rgb}{0.9,0,0}

\lstset{
    language=[Sharp]C,
    showspaces=false,
    showtabs=false,
    breaklines=true,
    frame=single,
    tabsize=3,
    showstringspaces=false,
    breakatwhitespace=true,
    escapeinside={(*@}{@*)},
    commentstyle=\color{greencomments},
    keywordstyle=\color{bluekeywords}\bfseries,
    stringstyle=\color{redstrings},
    basicstyle=\small\ttfamily,
    numbers=left
}

% barevné zvýraznění textů, které je nutno nahradit
\newcommand{\ZT}[1]{\colorbox{yellow}{\color{red}{#1}}}


% TOTO JE POTŘEBA ZMĚNIT !!!!!!
\newcommand{\nazevcz}{Vliv předzpracování obrazu a augmentace dat na segmentaci rentgenových snímků}        % zde VYPLŇTE český název práce (přesně podle zadání!)
\newcommand{\nazeven}{Impact of image preprocessing and data augmentation on segmentation of X-ray images}     % zde VYPLŇTE anglický název práce (přesně podle zadání!)
\newcommand{\autor}{Milan Gittler}           % zde VYPLŇTE své jméno a příjmení
\newcommand{\rok}{\the\year}                
\newcommand{\vedouci}{RNDr. Jíří Škvára, Ph.D.}         
% zde VYPLŇTE jméno a příjmení vedoucího práce, včetně titulů
\newcommand{\vedouciDAT}{RNDr. Jiřímu Škvárovi, Ph.D.}   
% zde VYPLŇTE jméno a příjmení vedoucího práce, včetně titulů ve třetím pádě
                                                           
% zvětšuje o 23% vertikální okraje v tabulkách
\renewcommand{\arraystretch}{1.23}

% nastavení pro záhlaví (co nelze udělat v cls souboru)
\renewcommand{\chaptermark}[1]{\markboth{\arabic{chapter}. #1}{}}
\pagestyle{fancy}

% nastavení odkazů
\usepackage{url} % formátování URL, příkaz \url
\usepackage{varioref} % lepší interní odkazy na obrázky, apod. příkaz \vref
\usepackage[unicode=true,pdfusetitle,
 bookmarks=true,
 breaklinks=false,pdfborder={0 0 1},backref=false,colorlinks=false]{hyperref} % hypertextové odkazy v PDF

\counterwithin{figure}{section}

\renewcommand{\listfigurename}{Seznam obrázků}
\renewcommand{\lstlistlistingname}{Sazba zdrojových kódů}

\begin{document}
\afterpage{\null\newpage}
\thispagestyle{empty}
\begin{center}
{
\LARGE
\univerzita\\[16pt]
\fakulta
}

\vspace{2cm}
\resizebox{8.42cm}{!}{%
\ifthenelse{\boolean{czech}}
{\includegraphics{Prilohy/Logo/LOGO_PRF_CZ_RGB_standard.jpg}}
{\includegraphics{Prilohy/Logo/LOGO_PRF_EN_RGB_standard.jpg}}}

\vspace{2cm}
{
\Huge\sffamily
\nazevcz\par
\vspace{0.6cm}
\Large\scshape \ifthenelse{\boolean{bc}}{bakalářská}{diplomová} práce
}
\end{center} 
 
\vfill
{
\large
\begin{tabular}{>{\bfseries}rl}
    Vypracoval: 	& \autor\\
    Vedoucí práce: 	& \vedouci\\
&\\
Studijní program:       & \program\\
% \ifthenelse{\boolean{api}}{Studijní obor:          & \obor\\}{}
\end{tabular} 
}
\vspace{1.5cm}
\begin{center}
  \Large\scshape   Ústí nad Labem \rok
\end{center}

\cleardoublepage
\thispagestyle{empty}
\pagecolor{yellow}
{\Large Namísto žlutých stránek vložte digitálně podepsané zadání kvalifikační práce poskytnuté vedoucím katedry.\\\
Zadání musí zaujímat právě dvě strany.
}

Zadání je nutno vložit jako PDF pomocí některého nástroje, který umožňuje editaci dokumentů (se zachováním
elektronického podpisu).

V Linuxe lze například použít příkaz \texttt{pdftk}.

\clearpage
\thispagestyle{empty}

\afterpage{\nopagecolor}
~
\clearpage

\thispagestyle{empty} 
{\bfseries Prohlášení}

\vspace{0.5cm}
Prohlašuji, že jsem tuto \ifthenelse{\boolean{bc}}{bakalářskou}{diplomovou} práci vypracoval\ifthenelse{\boolean{feminum}}{a}{}
samostatně a použil\ifthenelse{\boolean{feminum}}{a}{}
jen pramenů, které cituji a uvádím v přiloženém seznamu literatury.

\vspace{0.5em}

Byl\ifthenelse{\boolean{feminum}}{a}{} jsem seznámen\ifthenelse{\boolean{feminum}}{a}{} 
s tím, že se na moji práci vztahují práva a povinnosti vyplývající ze
zákona c. 121/2000 Sb., ve znění zákona c. 81/2005 Sb., autorský zákon, zejména se
skutečností, že Univerzita Jana Evangelisty Purkyně v Ústí nad Labem má právo na uzavření
licenční smlouvy o užití této práce jako školního díla podle § 60 odst. 1 autorského zákona, a
s tím, že pokud dojde k užití této práce mnou nebo bude poskytnuta licence o užití jinému
subjektu, je Univerzita Jana Evangelisty Purkyně v Ústí nad Labem oprávněna ode mne
požadovat přiměřený příspěvek na úhradu nákladu, které na vytvoření díla vynaložila, a to
podle okolností až do jejich skutečné výše.

\vspace{2em}

V Ústí nad Labem dne \today   \hfill Podpis: \makebox[4cm][s]{\dotfill}

\cleardoublepage
\thispagestyle{empty}
~
\vfill

\begin{flushright}
    Děkuji vedoucímu práce {\vedouciDAT}\\ 
    za neocenitelné rady a pomoc při tvorbě diplomové práce.
\end{flushright}

\cleardoublepage

\textsc{\nazevcz}

\textbf{Abstrakt:}

Hlavním cílem této diplomové práce je seznámit čtenáře s 

\textbf{Klíčová slova:} lorem, ipsum, dolor, sit, amet

\bigskip


\textsc{\nazeven}

\textbf{Abstract:}

lorem ipsum dolor sit amet

\textbf{Keywords:} lorem, ipsum, dolor, sit, amet

\tableofcontents

%-----------------------------------------------------------------------------%
	%\addchap{Úvod}
	%
	%\chapter{Historie}
	%    \section{Propaganda a dezinformace}
	%            \subsection{Propaganda}
	%            \subsection{Dezinformace}
	%                \subsubsection{Dezinformace %a misinformace}
	%     \section{Dezinformace napříč historií}
	     
	
% \chapter{Název kapitoly}
	    %section{Název sekce}
	    %   \begin{enumerate} % Číselný seznam
        %   \item Název prvního bodu
        %   \item Název druhého bodu
        %   \item Název třetího bodu
        %   \end{enumerate}
	    %       \subsection{Název sub-sekce}
	    %           \begin{itemize} % Bodový %seznam
                    % \item  Název bodu
                    % \item  Název bodu
                    % \item  Název bodu
                    % \end{itemize}
                     
   %\chapter{Název kapitoly s obrázkem}
       %Správný postup
       
       %\begin{figure}
       %    \centering
       %    \resizebox{10cm}{!}{\includegraphics{Prilohy/Logo/LOGO_PRF_CZ_RGB_standard.jpg}}
       %     \label{fig:logoPrF}
       %    \caption{Celé logo - Přirodovědecká fakulta}
       %\end{figure}
       %
       
       %%Horší postup
       
       %\begin{figure}[H]
       % \centering
       %    \includegraphics[width=10.5cm]{Prilohy/Logo/LOGO_PRF_CZ_RGB_standard.jpg}
       %    \label{fig:logoPrF}
       %    \caption{Celé logo - Přírodovědecká fakulta}
       %\end{figure}
       
       %Odkaz na obrázek v textu
       
       %\ref{fig:logoPrF}
        
    %\chapter{Název kapitoly s prvkem citace}
	%Pro odkaz na citaci stačí vložit příkaz %\texttt{cite} \cite{Katuscakc2008}.
	
	%\chapter{Závěr}
	
%-----------------------------------------------------------------------------%	

\addchap{Úvod}

\chapter{Přehled metod strojového učení}

\section{Počátky strojového učení}

\section{Základní pojmy}

\section{Algoritmy binární klasifikace}

\section{Klasifikace do více tříd}

\section{Metriky pro ohodnocení modelu}

\section{Knihovna ML.NET}

\section{Klasifikační úloha v ML.NET}

\section{Regresní úloha v ML.NET}

\subsection{Transformace dat a tvorba modelu}

\section{Model Builder}

\chapter{Ukázkové řešení úlohy intrusion detection}

\section{Popis úlohy}

\section{Načtení a transformace dat}

\section{Vizualizace dat}

\section{Tvorba modelů}

\section{Evaluace modelů}

\chapter{Zhodnocení} 

\chapter{Závěr}

\printbibliography[title=Seznam použitých zdrojů]

\listoffigures

\lstlistoflistings

\appendix

\chapter{Externí přílohy\label{sec:ep}}

%Na úložiští GitHub mohou byt uloženy tyto externí přílohy:

%\begin{itemize}
%\item \textbf{zdrojové kódy}
%\item \textbf{doplňkové texty} (například jak instalovat aplikaci, manuály aplikace)
%\item \textbf{schémata} (především, pokud se nevejdou na stranu A4 a jejich vytištění je tak problematické)
%\item \textbf{screenshoty} (v textu práce lze použít jen omezený počet snímků obrazovky, které navíc nemusí být při černobílém tisku příliš %přehledné)
%\item \textbf{videa} (například ovládání aplikace)
%\end{itemize}

%V každém případě by to však měli být pouze materiály, které jste vytvořili sami. Materiály jiných autorů uvádějte v seznamu použité %literatury (včetně případných odkazů na jejich originální umístění).

%V této kapitole stačí uvést pouze základní strukturu úložiště (co se kde nalézá a jakou má funkci) například v podobě tabulky. 
Struktura repozitáře je následující:
\begin{longtable}{ll}
\hline
BostonHousing & vypracovaná regresní úloha a data set Boston Housing \\
IrisFlowers & vypracovaná klasifikační úloha a data set Iris flowers \\
Obrázky & adresář s obrázky, které jsou zobrazeny v repozitáři \\
IntrusionDetection.rar & vypracovaná úloha Intrusion detection s data sety v souboru rar\\
README.md & jednoduchý popis repozitáře\\
\hline
\end{longtable}

%Všechny tyto soubory jsou potřeba pro překlad dokumentu (logo stačí jedno v příslušné jazykové verzi).

\end{document}
