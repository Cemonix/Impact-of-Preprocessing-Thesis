%volby: 
% male × female
% czech × english (zatím funguje jen czech)
% a studijní program / obor 
% is_bc (nejvíc odladěno)
% api_bc
% api_ing
% edu_bc
% edu_ing

\documentclass[male,czech,api_ing]{thesis}
\usepackage{ifthen}

\usepackage{amsmath,amssymb}
\usepackage{graphics}
\usepackage{color}
\usepackage{array}
\usepackage{longtable}
\usepackage{afterpage}
\usepackage{microtype}  % přesnější typografie
\usepackage{xcolor}
\usepackage{graphicx}

% Algorithms
\usepackage{algorithm}
\usepackage{algpseudocode}
\makeatletter
\renewcommand{\ALG@name}{Algoritmus}
\makeatother

\graphicspath{ {./images/} }

\usepackage{float}

% workaround for imcompatibility of czech babel and biblatex

\iftutex
\else
\usepackage{etoolbox}
\makeatletter
\newcommand\my@hyphen{-}
\newcommand\my@apostroph{'}
\patchcmd\select@language{-}{\my@hyphen }{}{}
\patchcmd\select@language{'}{\my@apostroph }{}{}
\makeatother
\fi

% fonty lze měnit (detaily viz sekce fonty)
\iftutex
	\usepackage{fontspec}  % nastavení fontů pro LuaLaTeX a XeLaTeX
	\setmainfont{Libertinus Serif}
	\setsansfont{Libertinus Sans}
	\setmonofont[Scale=MatchLowercase]{Libertinus Mono}
	\usepackage{unicode-math}
	\setmathfont{Libertinus Math}
\else
	\usepackage[utf8]{inputenc} % nastavení pro PDF LaTeX
	\usepackage[T1]{fontenc}
	\usepackage{libertinus}
	\renewcommand{\ttdefault}{pxtt}
\fi

\usepackage[style=iso-numeric,shortnumeration=true]{biblatex}
\addbibresource{thesis.bib}

\usepackage{csquotes} % uvozovky

% sazba ukázek kódu 

\usepackage{listings}
\renewcommand{\lstlistingname}{Ukázka kódu}

\lstdefinestyle{DOS}
{
    basicstyle=\scriptsize\color{black}\ttfamily,
    breaklines=true,
}

% ukázka pro nastavení balíku listings pro sazbu ukázek zdrojových kódů
\lstset{
  language=Python,                % the language of the code
  basicstyle=\small\ttfamily,    
  backgroundcolor=\color{white},   % choose the background color. You must add \usepackage{color}
  showspaces=false,                % show spaces adding particular underscores
  showstringspaces=true,           % underline spaces within strings
  showtabs=false,                  % show tabs within strings adding particular underscores
  frame=single,                    % adds a frame around the code
  tabsize=3,                       % sets default tabsize to 2 spaces
  breaklines=true,                 % sets automatic line breaking
  breakatwhitespace=false,         % sets if automatic breaks should only happen at whitespace
  keywordstyle=\bfseries,          % keyword style
  commentstyle=\rmfamily,       % comment style
  stringstyle=\itshape\color,   % string literal style
}

\definecolor{bluekeywords}{rgb}{0.13,0.13,1}
\definecolor{greencomments}{rgb}{0,0.5,0}
\definecolor{redstrings}{rgb}{0.9,0,0}

\lstset{
    language=[Sharp]C,
    showspaces=false,
    showtabs=false,
    breaklines=true,
    frame=single,
    tabsize=3,
    showstringspaces=false,
    breakatwhitespace=true,
    escapeinside={(*@}{@*)},
    commentstyle=\color{greencomments},
    keywordstyle=\color{bluekeywords}\bfseries,
    stringstyle=\color{redstrings},
    basicstyle=\small\ttfamily,
    numbers=left
}

% sazba matematických výrazů
\newenvironment{conditions}[1][kde:]
    {#1 \begin{tabular}[t]{>{$}l<{$} @{${}={}$} l}}
    {\end{tabular}\\[\belowdisplayskip]}

% barevné zvýraznění textů, které je nutno nahradit
\newcommand{\ZT}[1]{\colorbox{yellow}{\color{red}{#1}}}


% TOTO JE POTŘEBA ZMĚNIT !!!!!!
\newcommand{\nazevcz}{Vliv předzpracování obrazu a augmentace dat na segmentaci rentgenových snímků}        % zde VYPLŇTE český název práce (přesně podle zadání!)
\newcommand{\nazeven}{Impact of image preprocessing and data augmentation on segmentation of X-ray images}     % zde VYPLŇTE anglický název práce (přesně podle zadání!)
\newcommand{\autor}{Milan Gittler}           % zde VYPLŇTE své jméno a příjmení
\newcommand{\rok}{\the\year}                
\newcommand{\vedouci}{RNDr. Jíří Škvára, Ph.D.}         
% zde VYPLŇTE jméno a příjmení vedoucího práce, včetně titulů
\newcommand{\vedouciDAT}{RNDr. Jiřímu Škvárovi, Ph.D.}   
% zde VYPLŇTE jméno a příjmení vedoucího práce, včetně titulů ve třetím pádě
                                                           
% zvětšuje o 23% vertikální okraje v tabulkách
\renewcommand{\arraystretch}{1.23}

% nastavení pro záhlaví (co nelze udělat v cls souboru)
\renewcommand{\chaptermark}[1]{\markboth{\arabic{chapter}. #1}{}}
\pagestyle{fancy}

% nastavení odkazů
\usepackage{url} % formátování URL, příkaz \url
\usepackage{varioref} % lepší interní odkazy na obrázky, apod. příkaz \vref
\usepackage[unicode=true,pdfusetitle,
 bookmarks=true,
 breaklinks=false,pdfborder={0 0 1},backref=false,colorlinks=false]{hyperref} % hypertextové odkazy v PDF

\counterwithin{figure}{section}

\renewcommand{\listfigurename}{Seznam obrázků}
\renewcommand{\lstlistlistingname}{Sazba zdrojových kódů}

\begin{document}
\afterpage{\null\newpage}
\thispagestyle{empty}
\begin{center}
{
\LARGE
\univerzita\\[16pt]
\fakulta
}

\vspace{2cm}
\resizebox{8.42cm}{!}{%
\ifthenelse{\boolean{czech}}
{\includegraphics{Prilohy/Logo/LOGO_PRF_CZ_RGB_standard.jpg}}
{\includegraphics{Prilohy/Logo/LOGO_PRF_EN_RGB_standard.jpg}}}

\vspace{2cm}
{
\Huge\sffamily
\nazevcz\par
\vspace{0.6cm}
\Large\scshape \ifthenelse{\boolean{bc}}{bakalářská}{diplomová} práce
}
\end{center} 
 
\vfill
{
\large
\begin{tabular}{>{\bfseries}rl}
    Vypracoval: 	& \autor\\
    Vedoucí práce: 	& \vedouci\\
&\\
Studijní program:       & \program\\
% \ifthenelse{\boolean{api}}{Studijní obor:          & \obor\\}{}
\end{tabular} 
}
\vspace{1.5cm}
\begin{center}
  \Large\scshape   Ústí nad Labem \rok
\end{center}

\cleardoublepage
\thispagestyle{empty}
\pagecolor{yellow}
{\Large Namísto žlutých stránek vložte digitálně podepsané zadání kvalifikační práce poskytnuté vedoucím katedry.\\\
Zadání musí zaujímat právě dvě strany.
}

Zadání je nutno vložit jako PDF pomocí některého nástroje, který umožňuje editaci dokumentů (se zachováním
elektronického podpisu).

V Linuxe lze například použít příkaz \texttt{pdftk}.

\clearpage
\thispagestyle{empty}

\afterpage{\nopagecolor}
~
\clearpage

\thispagestyle{empty} 
{\bfseries Prohlášení}

\vspace{0.5cm}
Prohlašuji, že jsem tuto \ifthenelse{\boolean{bc}}{bakalářskou}{diplomovou} práci vypracoval\ifthenelse{\boolean{feminum}}{a}{}
samostatně a použil\ifthenelse{\boolean{feminum}}{a}{}
jen pramenů, které cituji a uvádím v přiloženém seznamu literatury.

\vspace{0.5em}

Byl\ifthenelse{\boolean{feminum}}{a}{} jsem seznámen\ifthenelse{\boolean{feminum}}{a}{} 
s tím, že se na moji práci vztahují práva a povinnosti vyplývající ze
zákona c. 121/2000 Sb., ve znění zákona c. 81/2005 Sb., autorský zákon, zejména se
skutečností, že Univerzita Jana Evangelisty Purkyně v Ústí nad Labem má právo na uzavření
licenční smlouvy o užití této práce jako školního díla podle § 60 odst. 1 autorského zákona, a
s tím, že pokud dojde k užití této práce mnou nebo bude poskytnuta licence o užití jinému
subjektu, je Univerzita Jana Evangelisty Purkyně v Ústí nad Labem oprávněna ode mne
požadovat přiměřený příspěvek na úhradu nákladu, které na vytvoření díla vynaložila, a to
podle okolností až do jejich skutečné výše.

\vspace{2em}

V Ústí nad Labem dne \today   \hfill Podpis: \makebox[4cm][s]{\dotfill}

\cleardoublepage
\thispagestyle{empty}
~
\vfill

\begin{flushright}
    Děkuji vedoucímu práce {\vedouciDAT}\\ 
    za neocenitelné rady a pomoc při tvorbě diplomové práce.
\end{flushright}

\cleardoublepage

\textsc{\nazevcz}

\textbf{Abstrakt:}

Hlavním cílem této diplomové práce je seznámit čtenáře s 

\textbf{Klíčová slova:} lorem, ipsum, dolor, sit, amet

\bigskip


\textsc{\nazeven}

\textbf{Abstract:}

lorem ipsum dolor sit amet

\textbf{Keywords:} lorem, ipsum, dolor, sit, amet

\tableofcontents

%-----------------------------------------------------------------------------%
	%\addchap{Úvod}
	%
	%\chapter{Historie}
	%    \section{Propaganda a dezinformace}
	%            \subsection{Propaganda}
	%            \subsection{Dezinformace}
	%                \subsubsection{Dezinformace %a misinformace}
	%     \section{Dezinformace napříč historií}
	     
	
% \chapter{Název kapitoly}
	    %section{Název sekce}
	    %   \begin{enumerate} % Číselný seznam
        %   \item Název prvního bodu
        %   \item Název druhého bodu
        %   \item Název třetího bodu
        %   \end{enumerate}
	    %       \subsection{Název sub-sekce}
	    %           \begin{itemize} % Bodový %seznam
                    % \item  Název bodu
                    % \item  Název bodu
                    % \item  Název bodu
                    % \end{itemize}
                     
   %\chapter{Název kapitoly s obrázkem}
       %Správný postup
       
       %\begin{figure}
       %    \centering
       %    \resizebox{10cm}{!}{\includegraphics{Prilohy/Logo/LOGO_PRF_CZ_RGB_standard.jpg}}
       %     \label{fig:logoPrF}
       %    \caption{Celé logo - Přirodovědecká fakulta}
       %\end{figure}
       %
       
       %%Horší postup
       
       %\begin{figure}[H]
       % \centering
       %    \includegraphics[width=10.5cm]{Prilohy/Logo/LOGO_PRF_CZ_RGB_standard.jpg}
       %    \label{fig:logoPrF}
       %    \caption{Celé logo - Přírodovědecká fakulta}
       %\end{figure}
       
       %Odkaz na obrázek v textu
       
       %\ref{fig:logoPrF}
        
    %\chapter{Název kapitoly s prvkem citace}
	%Pro odkaz na citaci stačí vložit příkaz %\texttt{cite} \cite{Katuscakc2008}.
	
	%\chapter{Závěr}
	
%-----------------------------------------------------------------------------%	

\addchap{Úvod}

\chapter{Přehled metod předzpracování obrazu}
V této kapitole se zaměříme na představení klíčových metod předzpracování obrazu, které hrají zásadní roli v procesu analýzy a zpracování rentgenových snímků. Předzpracování obrazu představuje kritický krok v řadě aplikací strojového učení a počítačového vidění, neboť ovlivňuje kvalitu a efektivitu následné analýzy. Metody předzpracování obrazu mohou výrazně zlepšit kvalitu dat a zvýšit přesnost detekce objektů, segmentace obrazu či klasifikace. Specifický výběr těchto technik je klíčový pro zjištění přesnosti a efektivity moderních metod počítačového vidění, včetně strojového učení a hlubokého učení, které jsou stále častěji aplikovány na širokou škálu problémů v oblasti zpracování obrazu. Zejména v kontextu rentgenových snímků může předzpracování pomoci překonat některé běžné výzvy, jako je nízký kontrast, šum, nebo artefakty, které mohou snížit kvalitu obrazu a tím ovlivnit diagnózu nebo automatickou analýzu.\cite{ImportanceOfImageProcessing}

\section{Klasické metody předzpracování obrazu}
Následující sekce se zabývá představením klasických metod a technik, které jsou využívány pro úpravu a zlepšení kvality obrazových dat před jejich dalším zpracováním. Budeme se tedy věnovat různým přístupům filtrace obrazu, metodám redukce šumu či zaostření obrazu a také technikám prahování a binarizace, které jsou fundamentální pro analýzu a interpretaci obrazových dat. Přestože se jedná o metody, které mohou být považovány za základní, jejich správná aplikace a kombinace mohou výrazně zlepšit výslednou kvalitu obrazu a přispět k efektivnějšímu rozpoznávání vzorů a objektů v obrazových datech. Podrobně prozkoumáme každou z těchto metod, přičemž budeme klást důraz na jejich význam pro přípravu dat k dalšímu zpracování.

\subsection{Vylepšení obrazu}
Vylepšení obrazu je klíčovou technikou v předzpracování obrazu, zaměřenou na zlepšení vizuální kvality obrazových dat pro následné zpracování nebo analýzu. Vylepšení obrazu usiluje o zlepšení kontrastu, jasu a ostrosti, aby bylo zajištěno, že obrazová data jsou co nejvíce přístupná pro lidské vnímání nebo automatizované algoritmy. 
Jedním z klíčů k úspěšnému zlepšení obrazu je výběr vhodné metody a jejích parametrů, které musí být pečlivě nastaveny v závislosti na charakteristikách obrazových dat a konkrétním účelu zpracování. 

\subsubsection{Ekvalizace histogramu}
Ekvalizace histogramu je fundamentální, ale mocná technika, která se používá pro zlepšení kontrastu v obrazových datech, zejména tam, kde původní obraz obsahuje špatně rozlišitelné detaily kvůli nedostatečnému rozsahu intenzit pixelů.

Cílem ekvalizace histogramu je aplikovat transformaci na původní histogram obrazu, $H(i)$, kde $i$ představuje intenzitu pixelů v původním obrazu, tak, aby výsledný histogram měl uniformní rozložení. Tato transformace je založena na kumulativní distribuční funkci (CDF), $CDF(i)$, vypočítané z původního histogramu. $CDF(i)$ představuje součet pravděpodobností všech intenzit pixelů od nejnižší hodnoty až po intenzitu $i$ a slouží jako mapovací funkce pro přiřazení nových intenzit pixelů ve výsledném obraze. Matematicky lze CDF definovat jako CDF(j) = $\sum_{i=0}^{j} P(i)$, kde $P(j)$ je pravděpodobnost výskytu intenzity $j$ v původním obrazu, což se obvykle určí normalizací histogramu na celkový počet pixelů v obrazu. Nová intenzita pixelu $i'$ pro každý pixel s původní intenzitou $i$ ve výsledném obrazu je poté určena pomocí normalizované CDF, což zajišťuje, že všechny intenzity jsou rovnoměrně zastoupeny:
\begin{equation}
    i' = (L-1) \cdot CDF(i)
\end{equation}
\begin{conditions}
    L &  je počet možných úrovní intenzity pixelů (např. $L = 256$ pro 8bitové obrazy)
\end{conditions}
Tímto způsobem transformace zvýší kontrast obrazu tak, že "roztáhne" rozložení intenzit pixelů přes celý dostupný rozsah, což zvýrazní detaily a zlepší vizuální vnímání obrazu. Ekvalizace histogramu tak přináší výrazné zlepšení v oblastech s nízkým kontrastem a umožňuje lepší vizualizaci detailů, což je zásadní pro dalšího zpracování obrazu. \cite{ComputerVisionMetrics} \cite{ComputerVisionForX-RayTesting}

\begin{algorithm}
    \caption{Ekvalizace histogramu}
    \begin{algorithmic}[1]
        \State \textbf{Vstup:} Původní obraz $Img$
        \State \textbf{Výstup:} Obraz s ekvalizovaným histogramem $Img_{eq}$
        \State Vypočítejte histogram $H$ z $Img$
        \State Vypočítejte kumulativní histogram $CH$ z $H$
        \State Normalizujte $CH$ na rozsah intenzit pixelů obrazu
        \For{každý pixel $p$ v $Img$}
            \State Nastavte $Img_{eq}[p]$ na hodnotu odpovídající $CH[Img[p]]$
        \EndFor
        \State \textbf{return} $Img_{eq}$
    \end{algorithmic}
\end{algorithm}

\subsubsection{Adaptivní ekvalizace histogramu}
Adaptivní ekvalizace histogramu (AHE) představuje pokročilou metodu ekvalizace, která se snaží zlepšit kontrast obrazu lokálně, na rozdíl od globálního přístupu klasické ekvalizace histogramu. AHE algoritmus rozdělí původní obraz na malé, překrývající se bloky, nazývané dlaždice (tiles), a na každou z nich aplikuje ekvalizaci histogramu nezávisle. Tímto způsobem dokáže lépe zachytit lokální kontrastní charakteristiky obrazu a zvýraznit detaily v jednotlivých oblastech. Při překryvu dlaždic se výsledné intenzity pixelů na okrajích vypočítají jako vážený průměr z odpovídajících intenzit získaných z každé příslušné dlaždice, což zajišťuje hladký přechod mezi dlaždicemi. \cite{GraphicsGems}

\subsubsection{Adaptivní ekvalizace histogramu s omezením kontrastu}
Adaptivní ekvalizace histogramu s omezením kontrastu (CLAHE) byla vyvinuta s cílem předejít problémům spojeným s přílišným zvýrazněním šumu v homogenních oblastech obrazu, které nastává při použití AHE.

Základní myšlenka CLAHE spočívá v rozdělení obrazu na malé, kontextově závislé bloky a aplikování ekvalizace histogramu na každý z těchto bloků. Aby se zabránilo nežádoucímu efektu nadměrného zvýšení kontrastu, aplikuje se na histogram každého bloku proces „osekání“ (clipping), kdy hodnoty histogramu přesahující předem definovaný limit jsou sníženy na tento limit a přebytečné hodnoty jsou rovnoměrně rozděleny mezi ostatní úrovně intenzity. To vede k vytvoření vyváženějšího rozložení intenzit v obrazu a zajišťuje, že zvýšení kontrastu nevede k nežádoucímu zvýraznění šumu. 

Jedním z klíčových aspektů CLAHE je výběr „clip limitu“, což je parametr, který omezuje míru zvýšení kontrastu. Tento limit se obvykle definuje jako násobek průměrné hodnoty histogramu a jeho správné nastavení je zásadní pro dosažení optimálního výsledku. \cite{GraphicsGems}

\subsubsection{Unsharp masking}
Unsharp masking je technika určená k zlepšení ostrosti obrazu tím, že zvýrazňuje hrany a detaily, které jsou v původním obrazu méně patrné. Klíčovým krokem této metody je vytvoření tzv. masky ostrých detailů odečtením rozmazané verze obrazu od jeho původní podoby. Rozmazaná verze obrazu se získává například pomocí Gaussova filtru, který je aplikován na původní obraz. Intenzita tohoto rozmazání určuje, jak silně budou hrany a detaily v maskě zvýrazněny. \cite{UnsharpMasking}

\begin{algorithm}
    \caption{Unsharp Masking}
    \begin{algorithmic}[1]
        \State \textbf{Vstup:} Původní obraz $I$, standardní odchylka Gaussova filtru $\sigma$, zesílení $\lambda$
        \State \textbf{Výstup:} Vylepšený obraz $I'$
        \State $G \gets \text{Gauss}(I, \sigma)$ \Comment{Vytvoření rozmazané verze obrazu pomocí Gaussova filtru}
        \State $M \gets I - G$ \Comment{Vytvoření masky ostrých detailů odečtením rozmazaného obrazu}
        \State $I' \gets I + \lambda M$ \Comment{Zvýšení ostrosti původního obrazu přičtením masky}
    \end{algorithmic}
\end{algorithm}

V tomto procesu:
\begin{itemize}
    \item $\sigma$ určuje míru rozmazání při použití Gaussova filtru.
    \item $\lambda$ je koeficient zesílení, který kontroluje míru, jakou jsou detaily zvýrazněny při přičítání masky k původnímu obrazu.
\end{itemize}

Při použítí unsharp masking metody může vznikat nežádoucí halo efektu. Tento efekt se projevuje jako nežádoucí světelný okraj kolem kontrastních hran, což může vést k umělému a nepřirozenému vzhledu obrazu. Integrace filtru zachovávajícího hrany (edge-preserving filter) do procesu unsharp masking je klíčová pro minimalizaci zmíněného efektu. Tento filtr umožňuje zvýraznit hrany a detaily v obrazu, aniž by docházelo k nežádoucím artefaktům. \cite{UnsharpMasking}

Unsharp masking metoda je vhodná pro aplikace, kde je žádoucí zlepšit viditelnost detailů a hran. Důležitým aspektem při použití unsharp masking je správný výběr parametrů $\sigma$ a $\lambda$, jelikož tyto hodnoty významně ovlivňují výsledný vzhled obrazu. Příliš vysoké hodnoty mohou vést k nežádoucím artefaktům, jako jsou halo efekty kolem hran, zatímco příliš nízké hodnoty mohou mít za následek nedostatečné zvýraznění detailů.

\subsection{Gama korekce}
Gamma korekce je technika používaná k úpravě jasu obrazových dat. Principem této metody je aplikace nelineární transformace na původní pixelové hodnoty obrazu, čímž se upravuje celková jasová křivka podle gamma funkce. Tato korekce umožňuje efektivněji využít dynamický rozsah zobrazovacích zařízení a zlepšit viditelnost detailů v tmavých i světlých oblastech obrazu bez ztráty informací v ostatních částech obrazového spektra. \cite{PreprocessingBook}

Aplikace gamma korekce se obvykle provádí podle vzorce:
\begin{equation}
    I' = I^\gamma
\end{equation}
\begin{conditions}
    I &  původní intenzita pixelu \\
    I' & nová intenzita pixelu po aplikaci gamma korekce \\
    \gamma & gamma faktor, který určuje míru korekce
\end{conditions}

Hodnota gamma faktoru menší než 1 zvýší jas tmavších oblastí obrazu, zatímco hodnota větší než 1 ztmaví světlejší oblasti a zvýší kontrast.

\subsubsection{Filtry pro zvýraznění ostrosti}

\subsection{Redukce šumu}

\subsection{Prahování a binarizace}

\section{Neuronové sítě pro předzpracování obrazu}

\subsection{Přehled a principy}

\section{Obecné postupy augmentace dat}

\chapter{Ukázkové řešení úlohy intrusion detection}

\section{Popis úlohy}

\section{Načtení a transformace dat}

\section{Vizualizace dat}

\section{Tvorba modelů}

\section{Evaluace modelů}

\chapter{Zhodnocení} 

\chapter{Závěr}

\printbibliography[title=Seznam použitých zdrojů]

\listoffigures

\lstlistoflistings

\appendix

\chapter{Externí přílohy\label{sec:ep}}

%Na úložiští GitHub mohou byt uloženy tyto externí přílohy:

%\begin{itemize}
%\item \textbf{zdrojové kódy}
%\item \textbf{doplňkové texty} (například jak instalovat aplikaci, manuály aplikace)
%\item \textbf{schémata} (především, pokud se nevejdou na stranu A4 a jejich vytištění je tak problematické)
%\item \textbf{screenshoty} (v textu práce lze použít jen omezený počet snímků obrazovky, které navíc nemusí být při černobílém tisku příliš %přehledné)
%\item \textbf{videa} (například ovládání aplikace)
%\end{itemize}

%V každém případě by to však měli být pouze materiály, které jste vytvořili sami. Materiály jiných autorů uvádějte v seznamu použité %literatury (včetně případných odkazů na jejich originální umístění).

%V této kapitole stačí uvést pouze základní strukturu úložiště (co se kde nalézá a jakou má funkci) například v podobě tabulky. 
Struktura repozitáře je následující:
\begin{longtable}{ll}
\hline
BostonHousing & vypracovaná regresní úloha a data set Boston Housing \\
IrisFlowers & vypracovaná klasifikační úloha a data set Iris flowers \\
Obrázky & adresář s obrázky, které jsou zobrazeny v repozitáři \\
IntrusionDetection.rar & vypracovaná úloha Intrusion detection s data sety v souboru rar\\
README.md & jednoduchý popis repozitáře\\
\hline
\end{longtable}

%Všechny tyto soubory jsou potřeba pro překlad dokumentu (logo stačí jedno v příslušné jazykové verzi).

\end{document}
